\documentclass[french]{article}
\usepackage[french]{babel}
\usepackage[T1]{fontenc}
\usepackage[utf8]{inputenc}
\usepackage{caption}
\usepackage[table]{xcolor}
\usepackage{float}
\usepackage{graphicx}
\usepackage[margin=1.50in]{geometry}
\usepackage[]{algorithm2e}
\usepackage{euler} 
\usepackage{xcolor}
\usepackage{mathtools}
\usepackage{amsmath, amsthm,amssymb}
\usepackage{listings}
\newcommand\mycommfont[1]{\footnotesize\ttfamily\textcolor{blue}{#1}}
\SetCommentSty{mycommfont}

\newcommand{\matr}[1]{\mathbf{#1}}
\def\frenchtablename {Tableau}
\mathtoolsset{showonlyrefs}
\title{TP3 : Modélisation et analyse de systèmes à l'aide des réseaux de Petri}
\author{
    JEAN-LANGLOIS Joël
    \and
    ST-AMOUR Francis
  }
\date{25 novembre 2016}

\begin{document}

\maketitle

\section{Introduction}

Les réseaux de pétri sont des outils mathématiques puissants qui permettent de modéliser des systèmes discrèts concurrents et de simuler leur comportement dynamique ainsi que de vérifier certaines propriétés de ce comportement.

\section{Exercices}

\subsection{Exercice 1}
Modèle de chemin de fer circulaire à deux train.

\subsubsection{Modélisez ce système par un réseau de Petri. Simulez votre modèle.}
Voir \textit{Ex\_1\_1.cpn} \\

Afin de modéliser le système décrit avec un réseau de prétrit non-coloré, nous avons suivit la suggestion de l'énnoncé qui était de modéliser chaque secteur par trois place \textit{O\textsubscript{ia}}, \textit{O\textsubscript{ib}} et \textit{E\textsubscript{i}} et de deux transitions \textit{Engager\textsubscript{ia}} et \textit{Engager\textsubscript{ib}.}  La simulation s'est déroulée comme prévu, sans bloquage et sans collision entre les trains. 
\subsubsection{Modélisez le même système par un réseau de Petri coloré. Simulez votre modèle. Comparez le avec celui obtenu 
en 1.}
Voir \textit{Ex\_1\_2.cpn} \\

L'ajout de couleur au réseau de pétri de la question suivante permet de diminuer de moitié les transitions ainsi que les places du style \textit{O\textsubscript{ix}}. Les connections sont plus clairs puisqu'il y en a moins et l'ajout de type permet de s'assurer que les connections suivent bien la sémantique voulue.
\subsubsection{Modélisez le même système par un réseau de Petri coloré à deux places O et E, et une seule transition 
\textit{Engagez-vous dans le secteur suivant}. Simulez votre modèle.}
Voir \textit{Ex\_1\_3.cpn}
\subsubsection{Indiquer les propriétés attendues de votre modèle et vérifier en utilisant l’outil.}
Les propriétés attendues sont les suivantes:
\begin{enumerate}
	\item Prop1
	\item Prop2
\end{enumerate}



\subsection{Exercice 2}
Protocole de communication entre deux processus.
\subsubsection{Modéliser le fonctionnement de ce protocole par un réseau de Petri. Montrer qu’il y a risque de 
blocage.}
Voir \textit{Ex\_2\_1.cpn}
\begin{proof}
Ce réseau possède un blocage puisque le marquage M' accessible à partir de $M_0$ sur la séquence de franchissements s = $t_1 
t_2 t_3$

\begin{align*}
M_0 &= \begin{pmatrix} 
	1 \\ 
	1 \\ 
	0 \\
	0 \\
	0 \\
	0   
\end{pmatrix} \xrightarrow{t_1 t_2 t_3} 
\begin{pmatrix} 
	0 \\ 
	0 \\ 
	0 \\
	0 \\
	1 \\
	1   
\end{pmatrix} = M'
\end{align*}
est un marquage puit, duquel aucune transition n'est tirable puisque M' > Pre(t) $\forall$ t $\in$ T.
\end{proof}
\subsubsection{Complétez le réseau précédent pour prendre en compte cette modification. Montrez que ce nouveau réseau est 
vivant.}
Voir \textit{Ex\_2\_2.cpn}
\begin{proof}
Le marquage initial de ce réseau est défini par $M_0(p1)$ = 1 > 0, $M_0(p2)$ = 1 > 0 et $M_0(p)$ = 0 pour toutes les autres 
places p. On pose la norme
\begin{align*}
\mu(M) = 2M(p_3) + 2M(p_4) + M(p_5) + M(p_6)
\end{align*}
À partir des invariants de place
\begin{align*}
M(p_1) + M(p_2) + M(p_3) + M(p_4) + M(p_5) + M(p_6) &= 2 \\
M(p_1) + M(p_3) + M(p_5) &= 1 \\
M(p_2) + M(p_4) + M(p_6) &= 1
\end{align*}
la vérification de la première condition de la norme $\mu(M)$ = 0 ssi M = $M_0$ est triviale. Les cas possibles pour 
$\mu$(M) > 0 et nécessairement M($p_1$) = M($p_2$) = 0 sont les 
suivants:
\begin{enumerate}
\item M($p_3$) > 0 et M($p_6$) >= 0: $t_2$ est tirable et $\mu$ décroît. Si M($p_6$) > 0, $t_6$ est tirable et $\mu$ décroît.
\item M($p_4$) > 0 et M($p_5$) >= 0: $t_3$ est tirable et $\mu$ décroît. Si M($p_5$) > 0, $t_4$ est tirable et $\mu$ décroît.
\item M($p_5$) > 0 et M($p_6$) > 0: $t_5$ est tirable et $\mu$ décroît. 	
\end{enumerate}
On a montré que la fonction $\mu$ est une norme pour le marquage initial $M_0$, celui-ci est donc un marquage d'accueil. De 
plus, le réseau est quasi-vivace car $\forall t \in T, \exists M \in R(N, M_0): M \xrightarrow{t}$
\begin{align*}
M_0 \xrightarrow{t_1} \\
M_0 \xrightarrow{t_1} M' \xrightarrow{t_2} \\
M_0 \xrightarrow{t_1} M' \xrightarrow{t_3} \\ 
M_0 \xrightarrow{t_1 t_2} M' \xrightarrow{t_4} \\ 
M_0 \xrightarrow{t_1 t_2 t_3} M' \xrightarrow{t_5} \\ 
M_0 \xrightarrow{t_1 t_3} M' \xrightarrow{t_6} 
\end{align*}
Le réseau est donc vivace.
\end{proof}
\subsubsection{Calculez, pour cette seconde version une base de l’espace des invariants de places. En déduire que le réseau 
est 1-borné.}
\begin{table}[h!]
	\centering
	\begin{tabular}{|>{\columncolor{lightgray}}c|c c c c c c|c c c|}
		\hline \rowcolor{lightgray}
		& $t_1$ & $t_2$ & $t_3$ & $t_4$ & $t_5$ & $t_6$ & $\vec{i_1}$ & $\vec{i_2}$ & $\vec{i_3}$ \\ \hline
		$p_1$ & -1 & 0 & 0 & 1 & 1 & 1 & 1 & 1 & 0   \\ \hline
		$p_2$ & -1 & 0 & 0 & 1 & 1 & 1 & 1 & 0 & 1   \\ \hline
		$p_3$ & 1 & -1 & 0 & 0 & 0 & -1 & 1 & 1 & 0   \\ \hline
		$p_4$ & 1 & 0 & -1 & -1 & 0 & 0 & 1 & 0 & 1   \\ \hline
		$p_5$ & 0 & 1 & 0 & -1 & -1 & 0 & 1 & 1 & 0   \\ \hline
		$p_6$ & 0 & 0 & 1 & 0 & -1 & -1 & 1 & 0 & 1   \\ \hline
	\end{tabular}
	\caption{Espace des invariants de places}
\end{table}
\begin{proof}
L'invariant $\vec{i_1}(p)$ = 1 > 0 $\forall$ p indique que le réseau est borné. On démontre à l'aide des invariants 
$\vec{i_2}$ et $\vec{i_3}$ que le réseau est 1-borné puisque l'ensemble des places le sont: 
\begin{align*}
M(p_i),i \in \{1,3,5\} \leq \frac{\vec{i_2} * M_0}{\vec{i_2}(p_i)} = (1, 0, 1, 0, 1, 0) * 
	\begin{pmatrix} 
	1 \\ 
	1 \\ 
	0 \\
	0 \\
	0 \\
	0   
	\end{pmatrix} = 1 \\
M(p_i),i \in \{2,4,6\} \leq \frac{\vec{i_3} * M_0}{\vec{i_3}(p_i)} = (0, 1, 0, 1, 0, 1) * 
\begin{pmatrix} 
1 \\ 
1 \\ 
0 \\
0 \\
0 \\
0   
\end{pmatrix} = 1 
\end{align*}
\end{proof}
\subsubsection{Modélisez le protocole dans sa deuxième version à l’aide d’un réseau coloré.}
Voir \textit{Ex\_2\_4.cpn}




\subsection{Exercice 3}
Modèle d'un Grand Prix F1
\subsubsection{Représentez par un réseau de Petri coloré l’avancement d’une voiture sur le circuit (il n’y a pas de 
limitation aux nombres de voitures qui peuvent occuper la même section). Vous préciserez les ensembles de couleurs que vous 
utilisez, le domaine de couleur des places et le marquage initial du réseau.}
Voir \textit{Ex\_3\_1.cpn}
\begin{flalign}
&Clr(p_1) = \{SECTION, VOITURE\}\quad donc\quad\{1..N-1\} x \{1..M\}& \\
&M_0 = \begin{pmatrix} 
<0, 1> + <0, 2> + ... + <0, M> \\
\end{pmatrix}&
\end{flalign}
\subsubsection{On considère à partir de maintenant qu’une fois le départ donné (toutes
les voitures ont quitté la section 0), il ne doit jamais y avoir plus d’une
voiture par section.Quelle contrainte cela introduit-il dans la règle d’avancement ? Modifiez votre modèle pour prendre en 
compte 
cette nouvelle contrainte.}
Voir \textit{Ex\_3\_2.cpn} \\
Cette nouvelle considération implique qu'il faudra que la transition ait accès à la voiture concernée par l'avancement, en 
plus d'avoir accès au contenu de la section suivante afin d'ajouter une garde sur la transition. Nous avons donc modifié 
notre paradigme pour que le RdPC reflète cette contrainte. Le domaine des couleurs pour la place $p_1$ est maintenant la 
produit cartésien du domaine des couleurs des SECTION avec le domaine de couleurs VOITURES d'une liste de jetons de la 
couleur 
VOITURE.
\begin{flalign}
  &Clr(p_1) = \{SECTION, VOITURES\}\cr
  &\quad avec\quad\{1..N-1\} x \{[VOITURE^{0..M}]_1,
[VOITURES^{0..M}]_2,...,[VOITURES^{0..M}]_{N-1}\}& \\
&M_0 = \begin{pmatrix} 
<0, [<v_1> + <v_2> + <v_3>+...+<v_M>]> + <1, []> + ... + <N-1, []> \\
\end{pmatrix}&
\end{flalign} 
De cette manière, il est possible de mettre une garde sur la transition de manière à ce qu'on puisse considérer les 
caractéristiques de la section suivante (et de sa liste de voiture(s)) simultanément à la section retenue. Il a ensuite 
suffit d'utiliser un bloc de code sur la transition pour sélectionner de façon non-déterministe la voiture retenue pour 
avancement dans la liste des voitures de la section courante et de la transférer à la section suivante:
\begin{lstlisting}
// guard
[ v1<>[] 
  andalso 
  List.null v2 
  andalso 
  if lastSection(sec1) then sec2=0 
  else sec2=sec1+1 ] 
  
// code segment
input (v1);
output(car_idx);
action
discrete (0, (length v1)-1)
\end{lstlisting}
\subsubsection{Lorsqu’une voiture se trouve sur la section 0, elle peut choisir de rentrer aux stands. Proposez une 
modélisation de cette action qui ne casse pas la structure circulaire du circuit (la section 0 se trouve toujours entre les 
sections N $-$ 1 et 1)}
Voir \textit{Ex\_3\_3.cpn}
\subsubsection{La sortie des stands se fait dans la section 1 et n’est possible que si cette section est vide. Ajoutez cette 
opération dans votre modèle.}
Voir \textit{Ex\_3\_4.cpn} \\
\subsubsection{Construisez la partie du modèle qui 
représente un dépassement.}
Voir \textit{Ex\_3\_5.cpn} \\
\subsubsection{Dépliez la partie du réseau qui représente le dépassement de Hamilton par Raïkkönen lorsque Hamilton est dans 
la section 3. Vous veillerez à donner des noms significatifs aux places du réseau.}
Voir \textit{Ex\_3\_6.cpn} \\
\section{Conclusion}
Finalement, 

\end{document}
